\documentclass[12pt, twocolumn]{article}   	% use "amsart" instead of "article" for AMSLaTeX format
\setlength{\columnsep}{0.75cm}
\usepackage[ margin=2cm]{geometry}                		% See geometry.pdf to learn the layout options. There are lots.
\geometry{letterpaper}                   		% ... or a4paper or a5paper or ... 
%\geometry{landscape}                		% Activate for for rotated page geometry
\usepackage[parfill]{parskip}    		% Activate to begin paragraphs with an empty line rather than an indent
\usepackage{graphicx}				% Use pdf, png, jpg, or eps§ with pdflatex; use eps in DVI mode
								% TeX will automatically convert eps --> pdf in pdflatex		
\usepackage{amssymb}

\title{Cohort model to solve for the duration of infectiousness distribution }
\author{David Champredon}
%\date{}							% Activate to display a given date or no date

\newcommand{\K}{\mathcal{K}}
\newcommand{\eqref}[1]{(\ref{#1})}


\begin{document}
\maketitle

Consider an $SIR$ model with non-linear recovery:

\begin{eqnarray}
S' & = & \mu S - \beta SI \\
I' & = & \beta S I - \gamma I \K(I) \label{eq:dI}
\end{eqnarray}

Note that equation \eqref{eq:dI} can be rewritten as a per-capita rate of prevalence change
\begin{equation}
\frac{I(t)'}{I(t)}  =  \beta S(t)  - \gamma \K(I(t)) \label{eq:dIcapita}
\end{equation}
where the time dependence ($t>0$) has been explicitly made.

In order to find the expression of the duration of infectiousness distribution, we consider the cohort of individuals who acquired the disease at exactly time $\alpha>0$. Let's label the size of this cohort (as a proportion of the whole population) $c_\alpha$. This cohort is depleted at the same per-capita rate as the one given in equation \eqref{eq:dIcapita}. Hence, if $c_\alpha(\tau)$ is the size of this cohort $\tau$ units of time after disease acquisition, we have:

\begin{equation}
\frac{c_\alpha(\tau)'}{c_\alpha(\tau)}  =   - \gamma \K(I(\alpha+\tau)) \label{eq:c_alpha}
\end{equation}

The initial condition for $c_\alpha$ is theoretically arbitrary, and if we choose $c_\alpha(0)=1$, we can interpret $c_\alpha(\tau)$ as the probability of still being infectious $\tau$ units of time after having acquired the disease at time $\alpha$, in other words, the duration of infectiousness distribution.

Let's express \eqref{eq:c_alpha} with the original time variable $t = \alpha+\tau$:

\begin{equation}
\frac{c_\alpha(t-\alpha)'}{c_\alpha(t-\alpha)}  =   - \gamma \K(I(t)) \label{eq:c_alpha_t}
\end{equation}
and for convenience, let's introduce $p$ as
$$p_\alpha(t) = c_\alpha(t-\alpha)$$


The duration of infectiousness distribution conditional on disease acquisition at time $\alpha$, $p_\alpha$, can be determined numerically by solving the following system:
\begin{eqnarray}
S' & = & \mu S - \beta SI \\
I' & = & \beta S I - \gamma I \K(I)\\ 
p_\alpha' & = & - \gamma p_\alpha\, \K(I) 
\end{eqnarray}
with the initial conditions $I(0) = i_0$, $S(0)=1-i_0$ and $p_\alpha(\alpha)=1$. Note that $p_\alpha$ is not epidemiologically defined for $t<\alpha$, but it is nonetheless possible to arbitrary set $p_\alpha(t)=1$ for $t\leq \alpha$.



\end{document}  